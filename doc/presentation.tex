\documentclass[10pt,a4paper]{beamer}
\usepackage[utf8]{inputenc}
\usepackage[french]{babel}
\usepackage[T1]{fontenc}
\usepackage{tikz}
%% \usepackage{graphicx}
%% \usepackage{lmodern}
%% \usepackage{kpfonts}
\author{Benjamin Bonneau \& Maxime Flin }
\title{Projet de système d'exploitation \\ --- Rock and Roll ---}
\begin{document}
\maketitle

\begin{frame}
  Sommaire:
  \begin{enumerate}
  \item Introduction
  \item Boot
  \item Mémoire
  \item Appels systèmes
  \item Processus
  \item Système de fichier
  \item Libc
  \item Conclusion et améliorations
  \end{enumerate}
\end{frame}

\begin{frame}
  \frametitle{Introduction}
  \textrm{ecos} est un système d'exploitation conçu pour une architecture Intel \textrm{x86} 64 bits. Il propose une interface très proche des normes \textrm{POSIX} et fourni
  \begin{itemize}
  \item une gestion de processus concurrent
  \item une gestion de plusieurs systèmes de fichiers
  \item une implémentation de la librarie standard C
  \item un shell avec des programmes utilitaires courants (\textrm{cat}, \textrm{ls}, etc\ldots)
  \end{itemize}
\end{frame}

\begin{frame}
  \begin{figure}
    \begin{tikzpicture}
      \draw[] (0, 6) rectangle (8, 7) node[pos=0.5] {programmes utilisateurs (\textrm{ls}, \textrm{cat}, \textrm{mat}, \textrm{edit}, \ldots)};
      \draw[] (4, 4.5) rectangle (6, 5.5) node[pos=0.5] {libc};

      \draw[->] (1, 5.8) -- (1, 4.1) node[right, pos=0.5] {appels systèmes};
      \draw[->] (5, 5.9) -- (5, 5.6);
      \draw[->] (5, 4.4) -- (5, 4.1);

      \draw[] (0, 1) rectangle (8, 4) node[above] {kernel};
      \draw[] (.1, 3) rectangle (3.9, 3.9) node[pos=0.5] { gestion des interruptions };
      \draw[] (4.1, 1.1) rectangle (7.9, 3.9) node[pos=0.5] { ordonnanceur };
      \draw[] (.1, 2.1) rectangle (3.9, 2.9) node[pos=.5] { vfs };
      \draw[] (.1, 1.1) rectangle (2, 2) node[pos=.5] { ext2 };
      \draw[] (2.1, 1.1) rectangle (3.9, 2) node[pos=.5] { procfs };

      \draw[->] (3.9, 3.5) -- (4.2, 3.5);

      \draw[->] (1.9, 3.1) -- (1.9, 2.8);
      \draw[->] (2.1, 2.8) -- (2.1, 3.1);

      \draw[->] (.9, 2.3) -- (.9, 1.8);
      \draw[->] (1.1, 1.8) -- (1.1, 2.3);

      \draw[->] (2.9, 2.3) -- (2.9, 1.8);
      \draw[->] (3.1, 1.8) -- (3.1, 2.3);

      \draw[->] (7, 4.1) -- (7, 5.8) node[right, pos=.5] { interruptions };

      \draw[] (0, 0) rectangle (3.9, .5) node[pos=.5] { clavier };
      \draw[] (4.1, 0) rectangle (8, .5) node[pos=.5] { horloge };

      \draw[->] (2, 0.6) -- (2, 0.85);
      \draw[->] (6, 0.6) -- (6, 0.85);
      \draw[] (4,.7) node { interruptions };
    \end{tikzpicture}
    \caption{une vue globale du système}
  \end{figure}
\end{frame}

\begin{frame}
  \frametitle{Processus}

  On maintient une tables de processus dans laquelle on sauvegarde pour chaque pid toutes les informations dont on pourrait avoir besoin:

  \begin{itemize}
  \item le status du processus associé (\textrm{Libre}, \textrm{Zombie}, \textrm{Endormi}, \textrm{Run}, \ldots)
  \item l'état des registres lors de la dernière interruption
  \item l'identifiant du paging
  \item la table des descripteurs de fichiers
  \end{itemize}

  Pour éviter d'avoir à parcourir toute la table on a ajouté aux processus une structure de liste chainée.

  \begin{figure}
    \begin{center}
      \begin{tikzpicture}
        \draw[] (0,0) rectangle (2, 1) node[pos=.5] {\textrm{pid} 1};
        \draw[] (2,0) rectangle (4, 1) node[pos=.5] {\textrm{pid} 2};
        \draw[] (4,0) rectangle (6, 1) node[pos=.5] {\textrm{pid} 3};
        \draw[] (6,0) rectangle (8, 1) node[pos=.5] {\textrm{pid} 4};
        \draw[] (8,0) rectangle (10, 1) node[pos=.5] {\textrm{pid} 5};
        \draw[] (10,0) -- (11, 0);
        \draw[] (10, 1) -- (11, 1);
        \draw[] (11, .5) node {\ldots};

        \draw[] (0.5, 1) edge[out=60,in=120,->] (2, 1);
        \draw[] (2.5, 1) edge[out=60,in=120,->] (6, 1);
        \draw[] (6.5, 1) edge[out=60,in=120,->] (10, 1);
        \draw[] (10.5, 1) edge[out=60,in=120,->] (14, 1);

        \draw[] (4.2, 0) edge[out=-60,in=-120,->] (8, 0);
        \draw[] (8.2, 0) edge[out=-60,in=-120,->] (12, 0);
      \end{tikzpicture}
      \caption{Table de processus avec la liste des enfants de 1 (en haut) et la liste des pid libres (en bas)}
    \end{center}
  \end{figure}
\end{frame}

\begin{frame}
  \frametitle{Ordonnacement des processus}
  On utilise une file de priorité pour classer les processus en fonction de leurs états. Dans la file, on trouve tous les processus qui attente de pouvoir retravailler.

  Le noyau reçoit régulièrement des interruptions du PIT\footnote{Programmable Interval Timer} qui interrompent le processus en cours. Si celui-ci travaille déjà depuis plus d'une certaine unité de temps\footnote{définie dans kernel/params.h}, alors on laisse travailler un autre processus. De cette manière, on évite les famines.

  On distingue deux processus particulers:
  \begin{itemize}
  \item le processus \textrm{init}, de pid 1, est le premier processus lancé par le noyau au moment du boot. Il lance un shell et attend que celui-ci termine pour le relancer.
  \item le processus \textrm{stop}, de pid 2, est utilisé en cas de panique du noyau.
  \end{itemize}

\end{frame}

\begin{frame}
  \frametitle{Système de fichier}

  On n'interragit avec aucune device réelle, par manque de temps, nous avons décidé de simplement simuler les comportement d'un tel composant. Lors de la compilation, nous générons un fichier binaire qui représente le système de fichier et le chargeons directement dans le binaire du noyau.


  Néamoins nous avons la possibilité d'interagir avec plusieurs devices en même temps. Pour chacune de ces devices, on précise
  \begin{itemize}
  \item sa position dans le système de fichier parent (si ce dernier existe)
  \item le système de fichier à utiliser pour lire dans cette device
  \item des informations spécifique au point de montage.
  \end{itemize}
\end{frame}

\begin{frame}
  \frametitle{Extended Filesystem 2}
  Un système de fichier développé dans les années 90 pour remplacer Minix, le système de fichier supporté par les premières versions de Linux.

  Les données sont réparties groupes de blocks pour essayer de maintenir les blocs d'un même fichiers adjacent.

  \begin{figure}
    \begin{center}
      \begin{tikzpicture}
        % inode
        \draw[] (0,3) rectangle (1, 4.5) node[pos=.5] {infos};
        \foreach \i in {1,...,15}
        \draw (0,\i * .2) rectangle (1, \i * .2 + .2);

        % 1
        \draw (2, 3.6) rectangle (3, 4.6);
        \draw (2, 2.3) rectangle (3, 3.3);
        \draw[->] (1, 2.85) -- (2, 4.6);
        \draw[->] (1, 2.25) -- (2, 3.3);

        \foreach \i in {1,...,4}
        \draw (2,\i * .2 + 1) rectangle (3, \i * .2 + .2 + 1);
        \draw[->] (1, .65) -- (2, 2);

        \foreach \i in {1,...,4}
        \draw (2,\i * .2) rectangle (3, \i * .2 + .2);
        \draw[->] (1, .45) -- (2, 1);

        % 2
        \draw (4, 3.8) rectangle (5, 4.8);
        \draw (4, 2.5) rectangle (5, 3.5);
        \draw (4, 1.3) rectangle (5, 2.3);
        \draw[->] (3, 1.85) -- (4, 4.8);
        \draw[->] (3, 1.65) -- (4, 3.5);
        \draw[->] (3, 1.45) -- (4, 2.3);

        \foreach \i in {1,...,4}
        \draw (4,\i * .2) rectangle (5, \i * .2 + .2);

        % 3
        \draw (6, 4) rectangle (7, 5);
        \draw (6, 2.7) rectangle (7, 3.7);
        \draw (6, 1.5) rectangle (7, 2.5);
        \draw (6, .2) rectangle (7, 1.2);

      \end{tikzpicture}
      \caption{représentation d'un inoeud}
    \end{center}
  \end{figure}

\end{frame}

\begin{frame}
  \frametitle{procfs}

  \textrm{procfs} est un système de fichier virtuel qui permet d'obtenir des informations sur le système à travers une interface classique de fichiers. Notre implémentation de ce système de fichier est spécifique à notre système. Il se structure de la manière suivante:

  \begin{itemize}
  \item tty
  \item pipes
  \item null
  \item display
  \item un dossier \textrm{pid} pour chaque processus vivant
    \begin{itemize}
    \item stat
    \item cmd
    \item fd
    \end{itemize}
  \end{itemize}

  Cette interface est très pratique pour développer des outils comme \textrm{ps} ou \textrm{lsof} qui peuvent ainsi facilement accéder aux données du kernel sans avoir à ajouter des appels systèmes complexes.
\end{frame}

\begin{frame}
  \frametitle{Système de fichier virtuel}

  Pour que l'utilisateur puisse manipuler tous les fichiers du systèmes à travers une unique interface, on ajoute une couche d'indirection entre lui et les systèmes de fichiers.

  C'est le système de fichier virtuel qui gère la hyérarchie entre les systèmes de fichier. Par exemple, dans notre démonstration \textrm{/} est une partition \textrm{ext2} qui contient un dossier \textrm{/bin} et deux autres systèmes de fichiers: \textrm{/proc} et \textrm{/home}.

  On maintient une table de \textit{fichiers virtuels} qui représentent les fichiers ouverts par le système. On y conserve le numéro d'inoeud du fichier et la device à laquell il appartient.

  Le même fichier peut être ouvert plusieurs fois a des endroits différents. Pour représenter ceci, les processus n'agissent pas directement sur des fichiers virtuels mais sur des channels qui référencent un fichier virtuel, un mode (écriture, lecture ou les deux) et une position dans le fichier.
\end{frame}

\begin{frame}
  \begin{figure}
    \caption{Les différentes couches d'indirections entre le fichier et le processus}
  \end{figure}
\end{frame}

\begin{frame}
  \frametitle{Libc et Libk}

\end{frame}

\begin{frame}
  \frametitle{Libc}
  Dans la libc, on trouve
  \begin{itemize}
  \item des fonctions utilitaires (\textrm{strlen}, \textrm{malloc}\ldots)
  \item des abstractions au dessus des appels systèmes (\textrm{opendir}, \textrm{fopen}, \ldots)
  \item les appels systèmes, qui se sont plus que des \textrm{int x80} précédé d'un \textrm{mov} de l'identifiant de de l'appel dans \textrm{\%rax}.
  \end{itemize}

  Il y a également une séquence de controle qui permet d'initialiser la libc, de commencer son programme dans la fonction \textrm{main} et d'\textrm{exit} le programme au retour  de \textrm{main}.

  Pour éviter de dupliquer le code de la libc dans les binaires chargées dans le système de fichier, on la charge dynamiquement quand on lance le processus.

  De plus, grâce aux interfaces les plus communes de la libc, du code générique respectant les normes \textrm{POSIX} pourrait être executé sur notre système.

\end{frame}

\end{document}
