\documentclass[10pt,a4paper]{beamer}
\usepackage[utf8]{inputenc}
\usepackage[french]{babel}
\usepackage[T1]{fontenc}
\usepackage{tikz}
%% \usepackage{graphicx}
%% \usepackage{lmodern}
%% \usepackage{kpfonts}
\author{Benjamin Bonneau \& Maxime Flin }
\title{Projet de système d'exploitation \\ --- Rock and Roll ---}
\begin{document}
\maketitle

\begin{frame}
  Sommaire:
  \begin{enumerate}
  \item Introduction
  \item Boot
  \item Mémoire
  \item Appels systèmes
  \item Processus
  \item Système de fichier
  \item Libc
  \item Conclusion et améliorations
  \end{enumerate}
\end{frame}

\begin{frame}
  \frametitle{Introduction}
  \textrm{ecos} est un système d'exploitation conçu pour une architecture Intel \textrm{x86} 64 bits. Il propose une interface très proche des normes \textrm{POSIX} et fourni
  \begin{itemize}
  \item une gestion de processus concurrent
  \item une gestion de plusieurs systèmes de fichiers
  \item une implémentation de la librarie standard C
  \item un shell avec des programmes utilitaires courants (\textrm{cat}, \textrm{ls}, etc\ldots)
  \end{itemize}
\end{frame}

\begin{frame}
  \begin{figure}
    \begin{tikzpicture}
      \draw[] (0, 6) rectangle (8, 7) node[pos=0.5] {programmes utilisateur (\textrm{ls}, \textrm{cat}, \textrm{mat}, \textrm{edit}, \ldots)};
      \draw[] (4, 4.5) rectangle (6, 5.5) node[pos=0.5] {libc};

      \draw[->] (1, 5.8) -- (1, 4.1) node[right, pos=0.5] {appels systèmes};
      \draw[->] (5, 5.9) -- (5, 5.6);
      \draw[->] (5, 4.4) -- (5, 4.1);

      \draw[] (0, 1) rectangle (8, 4) node[above] {kernel};
      \draw[] (.1, 3) rectangle (3.9, 3.9) node[pos=0.5] { gestion des interruptions };
      \draw[] (4.1, 1.1) rectangle (7.9, 3.9) node[pos=0.5] { ordonnanceur };
      \draw[] (.1, 2.1) rectangle (3.9, 2.9) node[pos=.5] { vfs };
      \draw[] (.1, 1.1) rectangle (2, 2) node[pos=.5] { ext2 };
      \draw[] (2.1, 1.1) rectangle (3.9, 2) node[pos=.5] { procfs };

      \draw[->] (3.9, 3.5) -- (4.2, 3.5);

      \draw[->] (1.9, 3.1) -- (1.9, 2.8);
      \draw[->] (2.1, 2.8) -- (2.1, 3.1);

      \draw[->] (.9, 2.3) -- (.9, 1.8);
      \draw[->] (1.1, 1.8) -- (1.1, 2.3);

      \draw[->] (2.9, 2.3) -- (2.9, 1.8);
      \draw[->] (3.1, 1.8) -- (3.1, 2.3);

      \draw[->] (7, 4.1) -- (7, 5.8) node[right, pos=.5] { interruptions };

      \draw[] (0, 0) rectangle (3.9, .5) node[pos=.5] { clavier };
      \draw[] (4.1, 0) rectangle (8, .5) node[pos=.5] { horloge }

      \draw[->] (2, 0.6) -- (2, 0.85);
      \draw[->] (6, 0.6) -- (6, 0.85);
      \draw[] (4,.7) node[center] { interruptions };

    \end{tikzpicture}
    \caption{une vue globale du système}
  \end{figure}
\end{frame}

\end{document}
