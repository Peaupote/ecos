\documentclass[10pt,a4paper]{beamer}
\usepackage[utf8]{inputenc}
\usepackage[french]{babel}
\usepackage[T1]{fontenc}
%% \usepackage{graphicx}
%% \usepackage{lmodern}
%% \usepackage{kpfonts}
\author{Benjamin Bonneau \& Maxime Flin }
\title{Projet de système d'exploitation \\ --- Rock and Roll ---}
\begin{document}
\maketitle

\begin{frame}
  Sommaire:
  \begin{enumerate}
  \item Introduction
  \item Boot
  \item Mémoire
  \item Appels systèmes
  \item Processus
  \item Système de fichier
  \item Libc
  \item Conclusion et améliorations
  \end{enumerate}
\end{frame}

\begin{frame}
  \frametitle{Introduction}
  \textrm{ecos} est un système d'exploitation. Il propose une interface très proche des normes \textrm{POSIX}. Il propose:
  \begin{itemize}
  \item une gestion de processus concurrent
  \item un système de fichier
  \item une implémentation de la librarie standard C
  \item des programmes utilitaires courants (\textrm{cat}, \textrm{ls}, etc\ldots)
  \item un shell
  \end{itemize}
\end{frame}

\end{document}
